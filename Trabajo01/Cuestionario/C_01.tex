\documentclass[a4paper, 11pt]{article}
\usepackage[cuestionario=1]{estilo}

\begin{document}

    \maketitle

    \section{Ejercicios}

    \begin{ejercicio}
        Identificar, para cada una de las siguientes tareas, qué tipo de aprendizaje es el adecuado (supervisado, no supervisado, por refuerzo) y los datos de aprendizaje que deberíamos usar. Si una tarea se ajusta a más de un tipo, explicar cómo y describir los datos para cada tipo.
        \begin{itemize}
            \item Categorizar un grupo de animales vertebrados en pájaros, mamíferos, reptiles, aves y anfibios.
            \item Clasificación automática de cartas por distrito postal.
            \item Decidir si un determinado índice del mercado de valores subirá o bajará dentro de un periodo de tiempo fijado.
        \end{itemize}
    \end{ejercicio}

    \begin{solucion}
        Supervisado.

        Supervisado.

        Refuerzo.
    \end{solucion}


    \begin{ejercicio}
        ¿Cuáles de los siguientes problemas son más adecuados para una aproximación por aprendizaje y cuáles más adecuados para una aproximación por diseño? Justificar la decisión.
        \begin{itemize}
            \item Determinar el ciclo óptimo para las luces de los semáforos en un cruce con mucho tráfico.
            \item Determinar los ingresos medios de una persona a partir de sus datos de nivel de educación, edad, experiencia y estatus social.
            \item Determinar si se debe aplicar una campaña de vacunación contra una enfermedad.
        \end{itemize}
    \end{ejercicio}

    \begin{solucion}
        Diseño

        Aprendizaje

        Aprendizaje
    \end{solucion}

    \begin{ejercicio}
        Construir un problema de aprendizaje desde datos para un problema de selección de fruta en una explotación agraria (ver transparencias de clase). Identificar y describir cada uno de sus elementos formales. Justificar las decisiones.
    \end{ejercicio}

    \begin{ejercicio}
        Suponga un modelo PLA y un dato $x(t)$ mal clasificado respecto de dicho modelo. Probar que la regla de adaptación de pesos del PLA es un movimiento en la dirección correcta para clasificar bien $x(t)$.
    \end{ejercicio}

    \begin{solucion}\footnote{Ejercicio resuelto con la ayuda de los pasos explicados en el Ejercicio 1.3 de \cite{mostafa2012learning}.}

        Sea $(x(t), y(t))$, con $y(t) \in \{-1, +1\}$ la muestra mal clasificada respecto del modelo PLA; esto es, $sign(w^T(t) x(t)) \neq y(t)$ o, dicho de otra manera
        \[
        y(t) sign(w^T(t) x(t)) < 0
        \]

        Probar que la regla de adaptación de pesos del PLA es un movimiento en la dirección correcta para clasificar bien $x(t)$ consiste en determinar que la siguiente iteración del algoritmo estará más cerca de clasificar bien la muestra; esto es, tenemos que probar lo siguiente:
        \begin{equation}
            y(t) sign(w^T(t+1) x(t)) > y(t) sign(w^T(t) x(t))
            \label{eq:PLA}
        \end{equation}
        donde $w^T(t+1)$ es el vector de pesos que nos da la siguiente iteración del algoritmo, y que viene determinado como
        \[
        w(t+1) = w(t) + y(t)x(t)
        \]

        Para probar la desigualdad \ref{eq:PLA} basta trabajar con la parte izquierda y la definición de $w^T(t+1)$ de la siguiente manera:
        \begin{align*}
            y(t) w^T(t+1) x(t) &= y(t) (w^T(t) + y(t)x(t)) x(t) = \\
            &= (y(t)w^T(t) + x(t)) x(t) = \\
            &= y(t)w^T(t)x(t) + x^2(t) > y(t)w^T(t)x(t)
        \end{align*}
        donde hemos usado que $y(t)^2 = 1$, pues $y(t) \in \{-1, +1\}$ y que $x^2(t) > 0$.

        Hemos probado así que la regla del PLA permite acercarse a la correcta etiquetación de la muestra.
    \end{solucion}


    \begin{ejercicio}
        Considere el enunciado del ejercicio 2 de la sección FACTIBILIDAD DEL APRENDIZAJE de la relación apoyo.
        \begin{itemize}
            \item Si $p = 0.9$, ¿cuál es la probabilidad de que S produzca una hipótesis mejor que C?
            \item ¿Existe un valor de $p$ para el cual es más probable que C produzca una hipótesis mejor que S?
        \end{itemize}
    \end{ejercicio}


    \begin{ejercicio}
        La desigualdad de Hoeffding modificada nos da una forma de caracterizar el error de generalización con una cota probabilística
        \[
        \mathbb{P}[\vert E_{out}(g) - E_{in}(g) \vert > \varepsilon] \leq 2 M e^{-2 N^2 \varepsilon}
        \]
        para cualquier $\varepsilon > 0$. Si fijamos $\varepsilon = 0.05$ y queremos que la cota probabilística $2 M e^{-2 N^2 \varepsilon}$ sea como máximo $0.03$, ¿cuál será el valor más pequeño de $N$ que verifique estas condiciones si $M = 1$? Repetir para $M = 10$ y para $M = 100$.
    \end{ejercicio}

    \begin{solucion}
        Si imponemos que la cota probabilística sea menor o igual que un valor $k$, basta despejar $N$ de la desigualdad
        \[
        2 M e^{-2 N^2 \varepsilon} \geq k
        \]
        y estudiar lo que se nos pide. Tenemos entonces:
        \begin{align*}
            2 M e^{-2 N \varepsilon^2} &\leq k \\
            e^{-2 N \varepsilon^2} &\leq \frac{k}{2M} \\
            -2 N \varepsilon^2 &\leq ln(\frac{k}{2M}) \\
            N &\geq \frac{1}{-2\varepsilon^2} ln(\frac{k}{2M})
        \end{align*}

        Es decir, la cota probabilística será menor o igual que $k$ si y sólo si $N \geq \sqrt{\frac{1}{-2\varepsilon} ln(\frac{k}{2M})}$. Como $N \in \mathbb{N}$, tenemos que el menor $N$ que cumple la condición para un $M$ dado es exactamente
        \begin{equation}
            N_M = \left\lceil \frac{1}{-2\varepsilon^2} ln\left(\frac{k}{2M}\right) \right\rceil
            \label{eq}
        \end{equation}
        donde $\lceil x \rceil$ es el menor entero mayor o igual que $x$.

        Ahora basta tomar $k = 0.03$, $\varepsilon = 0.05$ y, para $M \in \{1, 10, 100\}$, calcular la expresión obtenida en \ref{eq}, lo que nos da los siguientes valores:

        \begin{align*}
            N_1 &= 840 \\
            N_{10} &= 1301 \\
            N_{100} &= 1761
        \end{align*}
    \end{solucion}


    \begin{ejercicio}
        Consideremos el modelo de aprendizaje \guillemotleft M-intervalos \guillemotright donde $h \colon \mathbb{R} \to \{−1, +1\}$, y $h(x) = +1$ si el punto está dentro de cualquiera de $M$ intervalos arbitrariamente elegidos y −1 en otro caso. ¿Cuál es el más pequeño punto de ruptura para este conjunto de hipótesis?
    \end{ejercicio}

    \begin{solucion}
        Es claro que $M$ intervalos pueden separar cualquier muestra de $2M$ puntos. Imaginemos, por ejemplo, un tal conjunto de tamaño $2M$ en el que las muestras con distintas etiquetas se van alternando:
        \[
        +1 \;\;\; -1  \;\;\; +1  \;\;\; -1  \;\;\; \cdots  \;\;\; +1  \;\;\; -1
        \]

        Evidentemente, hay $M$ puntos etiquetados con $+1$, luego basta con \emph{rodear} esos puntos con los $M$ intervalos de los que disponemos para poder separar completamente la muestra.

        De hecho, esta dicotomía es la más \emph{difícil} de implementar, en el sentido de que necesitamos todos los intervalos disponibles para separarla. Supongamos ahora que intercambiamos dos puntos con etiquetas diferentes de la disposición anterior. En ese caso necesitaríamos sólo $M - 1$ intervalos para implementar la dicotomía. Cualquier otra disposición resulta en un número menor de intervalos necesarios, ya que agrupa puntos con etiquetas iguales bajo un mismo intervalo.

        Concluimos así que $m_{\mathcal{H}}(2M) = 2^{2M}$.

        Para ver que $2M + 1$ es un punto de ruptura basta probar que no hay ninguna muestra de ese tamaño de manera que $\mathcal{H}$ sea capaz de separar por completo.

        Consideramos de nuevo la misma muestra anterior, esta vez de tamaño $2M + 1$, para lo que añadimos un $+1$ al final; es decir, tenemos $M + 1$ puntos etiquetados como $+1$ y una dicotomía como la siguiente:
        \[
        +1 \;\;\; -1  \;\;\; +1  \;\;\; -1  \;\;\; \cdots  \;\;\; +1  \;\;\; -1  \;\;\; +1
        \]

        Si nuestro objetivo es implementar con $\mathcal{H}$ la muestra, debemos empezar por la izquierda y, cada vez que encontremos un $+1$, usar un intervalo para etiquetarlo bien. Esto no se puede hacer de otra manera, ya que los $+1$ tienen que estar dentro de un intervalo y los $-1$ fuera. Si seguimos hacia delante, habremos gastado los intervalos al \emph{encerrar} al $M$-ésimo $+1$, luego todo lo que haya a su derecha será etiquetado como $-1$, ya que se queda fuera de un intervalo. El último $+1$, por tanto, será etiquetado incorrectamente.

        Podríamos pensar que esto prueba que existe una única muestra que $\mathcal{H}$ no puede separar. Sin embargo, cualquier muestra de $2M + 1$ puntos en la recta real es de la forma descrita, y en todas y cada una de ellas existe al menos esta dicotomía que $\mathcal{H}$ no puede implementar.

         Tenemos así que cualquier muestra de $2M + 1$ puntos tiene al menos una dicotomía que $\mathcal{H}$ no puede implementar; es decir, $k = 2M + 1$ es un punto de ruptura.

        Entonces, como $m_{\mathcal{H}}(2M + 1) < 2^{2M}$ y $m_{\mathcal{H}}(2M) = 2^{2M}$, podemos ya afirmar que $k = 2M + 1$ es el más pequeño punto de ruptura para este conjunto de hipótesis.
    \end{solucion}


    \begin{ejercicio}
        Suponga un conjunto de $k^*$ puntos $x_1, x_2 , \dots , x_{k^*}$ sobre los cuales la clase $\mathcal{H}$ implementa $< 2^{k^*}$ dicotomías. ¿Cuáles de las siguientes afirmaciones son correctas?
        \begin{itemize}
            \item $k^*$ es un punto de ruptura.
            \item $k^*$ no es un punto de ruptura.
            \item Todos los puntos de ruptura son estrictamente mayores que $k^*$.
            \item Todos los puntos de ruptura son menores o iguales a $k^*$
            \item No conocemos nada acerca del punto de ruptura.
        \end{itemize}
    \end{ejercicio}

    \begin{solucion}
        En este caso no conocemos nada acerca del punto de ruptura. Por definición, $k$ es un punto de ruptura si $m_\mathcal{H}(k) < 2^k$. Esto no quiere decir que \emph{exista} una muestra de $k$ puntos para la que $H$ no sea capaz de implementar todas las dicotomías; sino que $\mathcal{H}$ es incapaz de implementar todas las dicotomías \emph{para todas} las muestras de $k$ puntos.
    \end{solucion}


    \begin{ejercicio}
        Para todo conjunto de $k^*$ puntos, $\mathcal{H}$ implementa $< 2^{k^*}$ dicotomías. ¿Cuáles de las siguientes afirmaciones son correctas?
        \begin{itemize}
            \item $k^*$ es un punto de ruptura.
            \item $k^*$ no es un punto de ruptura.
            \item Todos los $k \geq k^*$ son puntos de ruptura.
            \item Todos los $k < k^*$ son puntos de ruptura.
            \item No conocemos nada acerca del punto de ruptura.
        \end{itemize}
    \end{ejercicio}

    \begin{solucion}
        En este caso ya sí podemos afirmar, usando el mismo argumento que en el ejercicio anterior, que $k^*$ es un punto de ruptura.

        Por otro lado, es evidente que si $k^*$ es un punto de ruptura, $k \geq k^*$ también lo es. Esto es claro: si $\mathcal{H}$ es incapaz de separar una muestra de $k^*$ puntos, añadir puntos a la muestra no hará sino hacerla más compleja, luego $\mathcal{H}$ seguirá siendo incapaz de separarla.

        Por último, poco podemos decir sobre los $k < k^*$. Pueden ser puntos de ruptura, si se da el caso de que $k^*$ no sea el más pequeño punto de ruptura ---basta usar el mismo argumento que en el párrafo anterior---, o pueden no serlo, como en el caso sencillo del perceptron en el plano, donde sabemos que $k^* = 4$ es un punto de ruptura y $k = 3 < k^*$ no lo es.
    \end{solucion}



    \begin{ejercicio}
        Si queremos mostrar que $k^*$ es un punto de ruptura, ¿cuáles de las siguientes afirmaciones nos servirían para ello?:
        \begin{itemize}
            \item Mostrar que existe un conjunto de $k^*$ puntos $x_1, x_2 , \dots , x_{k^*}$ que $\mathcal{H}$ puede separar ---\emph{shatter}---.
            \item Mostrar que $\mathcal{H}$ puede separar cualquier conjunto de $k^*$ puntos.
            \item Mostrar un conjunto de $k^*$ puntos $x_1, x_2 , \dots , x_{k^*}$ que $\mathcal{H}$ no puede separar.
            \item Mostrar que $\mathcal{H}$ no puede separar ningún conjunto de $k^*$ puntos.
            \item Mostrar que $m_\mathcal{H} (k) = 2^{k^*}$
        \end{itemize}
    \end{ejercicio}

    \begin{solucion}
        Por definición, $k^*$ es un punto de ruptura si $m_\mathcal{H}(k^*) < 2^{k^*}$. Como la función de crecimiento está definida en términos del máximo número de dicotomías que $H$ puede implementar, podemos concluir que para probar que $k^*$ es un punto de ruptura hay que mostrar que $\mathcal{H}$ no puede separar \emph{ningún} conjunto de $k^*$ puntos.
    \end{solucion}


    \begin{ejercicio}
        Para un conjunto $\mathcal{H}$ con $d_{VC} = 10$, ¿qué tamaño muestral se necesita ---según la cota de generalización--- para tener un 95\% de confianza de que el error de generalización sea como mucho $0.05$?
    \end{ejercicio}

    \begin{ejercicio}
        Consideremos un escenario de aprendizaje simple. Supongamos que la dimensión de entrada es uno. Supongamos que la variable de entrada $x$ está uniformemente distribuida en el intervalo $[-1, 1]$ y el conjunto de datos consiste en 2 puntos $\{x_1, x_2\}$ y que la función objetivo es $f(x) = x^2$. Por tanto el conjunto de datos completo es $\mathcal{D} = \{(x_1 , x_1^2), (x_2, x_2^2)\}$. El algoritmo de aprendizaje devuelve la línea que ajusta estos dos puntos como $g$; es decir, $\mathcal{H}$ consiste en funciones de la forma $h(x) = ax + b$.
        \begin{itemize}
            \item Dar una expresión analítica para la función promedio $\bar{g}(x)$.
            \item Calcular analiticamente los valores de $E_{out}$, \emph{bias} y \emph{var}.
        \end{itemize}
    \end{ejercicio}


    \section{Bonus}

    \begin{bonus}
        Considere el enunciado del ejercicio 2 de la sección ERROR Y RUIDO de la relación de apoyo.
        \begin{itemize}
            \item Si su algoritmo busca la hipótesis $h$ que minimiza la suma de los valores absolutos de los errores de la muestra,
            \[
            E_{in}(h) = \sum_{n=1}^N \vert h - y_n \vert
            \]
            entonces mostrar que la estimación será la mediana de la muestra, $h_{med}$ ---cualquier valor que deje la mitad de la muestra a su derecha y la mitad a su izquierda---.
            \item Suponga que $y_N$ es modificado como $y_N + \varepsilon$, donde $\varepsilon \to \infty$. Obviamente el valor de $y_N$ se convierte en un punto muy alejado de su valor original. ¿Cómo afecta esto a los estimadores dados por $h_{mean}$ y $h_{med}$?
        \end{itemize}
    \end{bonus}



    \begin{bonus}
        Considere el ejercicio 12.
        \begin{itemize}
            \item Describir un experimento que podamos ejecutar para determinar ---numéricamente--- $\bar{g}(x)$, $E_{out}$, \emph{bias} y \emph{var}.
            \item Ejecutar el experimento y dar los resultados. Comparar $E_{out}$ con $bias+var$. Dibujar en unos mismos ejes $\bar{g}(x)$, $E_{out}$ y $f(x)$.
        \end{itemize}
    \end{bonus}

    \nocite{*}

% -------------------------- Bibliografía ------------------------
\bibliographystyle{babplain}
\bibliography{C_01}

\end{document}
