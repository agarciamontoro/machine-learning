\documentclass[a4paper, 11pt]{article}
\usepackage[cuestionario=1]{estilo}

\begin{document}

    \maketitle

    \section{Bonus}

    \begin{ejercicio}
        Resolver los siguientes problemas:
        \begin{enumerate}
            \item La distancia entre dos curvas en el plano está dada por el mínimo de la expresión $\sqrt{(x_1-x_2)^2+(y_1-y_2)^2}$ donde $(x_1,y_1)$ está sobre una de las curvas y $(x_2,y_2)$ está sobre la otra. Calcular la distancia entre la línea $x+y=4$ y la elipse $x^2+2y^2=1$.
        \end{enumerate}
    \end{ejercicio}

    \begin{solucion}
        Lo primero que tenemos que notar es que el mínimo de $g(x_1, y_1, x_2, y_2) =  \sqrt{(x_1-x_2)^2+(y_1-y_2)^2}$ se alcanza en el mismo punto que el mínimo de $g^2(x_1, y_1, x_2, y_2) =  (x_1-x_2)^2+(y_1-y_2)^2$. Tomando $g^2$ como función a minimizar y con las restricciones enunciadas, el lagrangiano con el que tenemos que trabajar es el siguiente:
        \[
        \mathcal{L}(x_1, y_1, x_2, y_2, \lambda, \mu) = (x_1-x_2)^2 + (y_1-y_2)^2 - \lambda (x_1+y_1-4) - \mu (x_2^2+2y_2^2-1)
        \]

        Calculamos sus derivadas parciales:
        \begin{align*}
            \frac{\partial}{\partial x_1} \mathcal{L} &= 2(x_1 - x_2) - \lambda \\
            \frac{\partial}{\partial y_1} \mathcal{L} &= 2(y_1 - y_2) - \lambda \\
            \frac{\partial}{\partial x_2} \mathcal{L} &= -2(x_1 - x_2) - 2\mu x_2 \\
            \frac{\partial}{\partial y_2} \mathcal{L} &= -2(y_1 - y_2) - 4\mu y_2 \\
            \frac{\partial}{\partial \lambda} \mathcal{L} &= x_1 + y_1 - 4 \\
            \frac{\partial}{\partial \mu} \mathcal{L} &= x_2^2 + 2y_2^2 - 1
        \end{align*}

        Igualando a cero las derivadas parciales anteriores y despejando los términos con $\lambda$ o $\mu$ llegamos al sistema de ecuaciones siguiente, cuyas soluciones son los puntos críticos del lagrangiano $\mathcal{L}$; es decir, los candidatos a mínimo:

        \begin{align}
            2(x_1 - x_2) &= \lambda \label{eq:lag1}\\
            2(y_1 - y_2) &= \lambda \label{eq:lag2}\\
            -2(x_1 - x_2) &= - 2\mu x_2 \label{eq:lag3}\\
            -2(y_1 - y_2) &= - 4\mu y_2 \label{eq:lag4}\\
            x_1 + y_1 &= 4 \label{eq:lag5}\\
            x_2^2 + 2y_2^2 &= 1 \label{eq:lag6}
        \end{align}

        Si ahoar sumamos las ecuaciones \ref{eq:lag1} y \ref{eq:lag3} por un lado y la \ref{eq:lag2} y \ref{eq:lag4} por otro llegamos a las expresiones siguientes:

        \begin{align}
            \lambda &= -2 \mu x_2 \label{eq:resol1} \\
            \lambda &= -4 \mu y_2 \label{eq:resol2}
        \end{align}

        Igualando \ref{eq:resol1} y \ref{eq:resol2}, y teniendo en cuenta que $\mu \neq 0$ ---esto es evidente, ya que si lo fuera, por \ref{eq:lag3} y \ref{eq:lag4} tendríamos un punto de intersección entre ambas curvas, situación que no se da---, llegamos a
        \[
        x_2 = 2y_2
        \]

        Imponiendo \ref{eq:lag6} concluimos que
        \begin{align}
            x_2 &= \sqrt{\frac{2}{3}} \label{eq:sol1} \\
            y_2 &= \frac{1}{2}\sqrt{\frac{2}{3}} = \frac{1}{\sqrt{6}} \label{eq:sol2}
        \end{align}

        Igualando ahora \ref{eq:lag1} y \ref{eq:lag2} y sustituyendo los valores que acabamos de deducir tenemos el siguiente par de ecuaciones junto con \ref{eq:lag5}:

        \begin{align*}
            x_1 - y_1 &= \sqrt{\frac{2}{3}} - \frac{1}{\sqrt{6}}
            x_1 + y_1 = 4
        \end{align*}
        de donde concluimos que
        \begin{align}
            x_1 &= 2 + \frac{1}{2\sqrt{6}} \\
            y_1 &= 2 - \frac{1}{2\sqrt{6}}
        \end{align}

        La distancia entre la recta y la elipse es, por tanto:

        \[
        g(2 + \frac{1}{2\sqrt{6}}, 2 - \frac{1}{2\sqrt{6}}, \sqrt{\frac{2}{3}}, \frac{1}{\sqrt{6}}) = \sqrt{2} \left(2 - \frac{3}{2\sqrt{6}}\right) \approx 1.962402
        \]

    \end{solucion}

\end{document}
